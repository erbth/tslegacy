% !TeX spellcheck = en_US
\documentclass[a4paper]{article}

\usepackage{tabularx}

\title{TSClient LEGACY Packaging Policy}
\author{Thomas Erbesdobler \textless t.erbesdobler@gmx.de\textgreater}

\begin{document}
	\maketitle
	\tableofcontents

	\section{ABI versioning and essential packages}
	
	\section{Notes about specific packages}
	
	This is the moment when I realize that it's gonna be so much more than I had believed ever. That I'll dig much deeper into each package than I thought, and that I'll give it much more time than I had ever meant.
	
	\subsection{util-linux}
	
	I decided to build util-linux without support for udev for now. I do not want to work around this cyclic dependency yet. We will see what will come.
	
	\subsection{elfutils}
	
	I decided to build elfutils without support for lzma. The two compression packages, zlib and bzip2, may suffice. However I am not sure about this yet, it is just trying.
	
	\section{Various packages}
	
	\begin{tabularx}{\textwidth}{l|X}
		Name & Description \\
		\hline
		tslegacy-compiletime & A bootable compiletime system. \\
		tslegacy-config & Configuration of less system-near services \\
		tslegacy-sysconfig & Basic system configuration \\
	\end{tabularx}

	\subsection{tslegacy-config vs. tslegacy-sysconfig}
	'tslegacy-sysconfig' contains system near configuration files like /etc/group or /etc/inittab. Therefore, many packages like perl or util-linux depend on it which makes changes expensive, since they trigger rebuild of the dependent packages according to the current package build model. 'tslegacy-config' tries to mitigate this by included less system near config files like /etc/profile, which are not essential for a system to provide basic operating system level services, but required for the whole system to work or for users' convenience. Hence, only the 'toplevel meta packages' like tslegacy shall depend on these.
\end{document}